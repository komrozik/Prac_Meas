%\newpage
\section{Execution}
\label{sec:execution}
\section{Components}
\subsection{Arduino Uno}
The Arduino
\subsection{Electrical Components}
\subsubsection{LED}
LED is short for Light Emitting Diode and therefore LED's are a special kind of Diode which emits light if a current is applied.
Diodes are electrical components whose conductivity depends on the applied pooling.
Classical semiconducter diodes are based on the physical properties of a pn-junction as explained in \cite{elektronik-kompendium}.
\subsubsection{Temperature Sensor DS18B20}
The DS18B20 is a programmable thermometer that can be controlled via a single cable.
Special advantages of the temperature sensor are that it requires only one cable for control besides the ground connection and due to the unique 64-bit serial codes of the sensors it is possible to control multiple sensors with a single cable.
In the range of \SI{-10}{\celsius} to \SI{85}{\celsius}, the temperature sensor has an accuracy of $\pm 0.5$\si{\celsius} and can be used in a temperature range from \SI{-55}{\celsius} to \SI{125}{\celsius}.
With the help of a prasitic power mode it is even possible to operate the temperature sensor with only 2 connections (GND and DQ) whereby the power supply is then also provided via the signal connection. In our case, however, we used 3 cables for operation.
Because the temperature sensor is connected to a 3-state port for its input signal, a pull-up resistor is used on the control line.

\subsection{Experiment}
The different Thermometers from the Arduino Circuits are connected to a handheld lab thermometer in a way that all thermometer tips align.
For the expirement a watertank with a heating/cooling plate and a pump to mix up the watercolumn is brought to a temperature of \SI{3}{\celsius} while the surronding air temperature is at \SI{25}{\celsius}.
To calibrate the sensors, all connected sensors are immersed in the water bath at the same time.
After a short time the temperature of the water bath is changed and increased to \SI{20}{\celsius}.
Each difference of \SI{1}{\kelvin} is noted with the corresponding timestamp.

\subsection{Arduino Coding}
Void Setup:
% \begin{lstlisting}[language=C++]
% void setup()
% {
%   Serial.begin(9600);

%   if(!rtc.begin()){
%     Serial.println("RTC is NOT running. Let's set the time now!");
%     rtc.adjust(DateTime(F(__DATE__), F(__TIME__)));
%   }
%   /******
%    * When the time needs to be set on a new device,
%    * or after a power loss,
%    * the following line sets the RTC to the date & time
%    * this sketch was compiled(!)
%   */
%   // rtc.adjust(DateTime(F(__DATE__), F(__TIME__)));
%   if(!SD.begin(10)){
%     Serial.println(
%         "SD module initialization failed
%         or SD Card is not present!"
%         );
%     return;
%   }

%   pinMode(5, OUTPUT);
%   pinMode(6, OUTPUT);
  
%   save_header();
% }
% \end{lstlisting}
Loop:
% \begin{lstlisting}[language=C++]
% void loop()
% {
%   byte rom_code[8]; // create array with 8 bytes (64 bits)
%   byte sp_data[9]; // Scratchpad data

%   //Start sequence to read out the rom code
%   ow.reset();
%   ow.write(READ_ROM);
%   for (int i = 0; i<8; i++) {
% rom_code[i] = ow.read();
%   }
%   if(rom_code[0] != IS_DS18B20_SENSOR){
%     Serial.print("Sensor is not a DS18B20 sensor!");
%     return;
%   }
%   String registration_number;
%   for (int i = 1; i<7;i++){
%     registration_number += String(rom_code[i], HEX);
%   }
  
%   // Start Sequence for converting temperatures

%   ow.reset();
%   ow.write(SKIP_ROM);
%   ow.write(CONVERT_T);

%   // Start sequence for reading data from scratchpad
%   ow.reset();
%   ow.write(SKIP_ROM);
%   ow.write(READ_SCRATCH);
%   for (int i = 0; i<9;i++){
%     sp_data[i] = ow.read();
%   }

%   int16_t tempRead = sp_data[1] << 8 | sp_data[0];
  
%   float tempCelsius = tempRead/16.;
%   // LED(int(tempCelsius));
%   String time_now= getISOtime();
%   save_data_point(time_now,registration_number,tempCelsius);
%   delay(1000);

% }
% \end{lstlisting}